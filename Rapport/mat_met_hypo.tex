\section{Matériel et Méthode}
\subsection{Matériel: \bsc{AR Drone 2.0}}
Nous allons étudier un \bsc{AR Drone 2.0}. C'est un hélicoptère quadrirotor pilotable via une liaison WiFi au travers d'une application disponible sous iOS, Android, Linux ou Windows. C'est un drone civil principalement destiné au divertissement. Il est équipé de:
\medbreak
\begin{itemize}
    \item un \textit{processeur ARM Cortex A8 32 bits cadencé à 1 GHz}
    \item \textit{1 Go de RAM DDR2 cadencée à 200 MHz}
    \item un \textit{système d'exploitation Linux 2.6.32}
    \item un \textit{module WiFi b/g/n}
    \item un \textit{accéléromètre 3 axes}
    \item un \textit{gyroscope 3 axes}
    \item un \textit{capteur de pression}
    \item un \textit{un magnétomètre 3 axes}
    \item des \textit{capteurs de proximité à ultrasons}
    \item une \textit{caméra vericale QVGA}
    \item un \textit{port USB 2.0}
\end{itemize}

\subsection{Méthode}


\newpage
\section{Hypothèse de travail}
Pour ce projet, nous supposerons que le drone utilisé est un ARDrone 2.0 qui n'a pas subi de modifications logicielles et qui est donc dans un état identique à sa sortie d'usine. Il dispose ainsi uniquement des protections éventuelles prévues par le constructeur Parrot. Nous supposons dans le cadre de ce projet que nous sommes  dans la position de l'attaquant et que l'ARDrone 2.0 est connecté et contrôlé par un client légitime.
