\section{Conclusion}
L'ARDrone 2.0 de \textbf{Parrot} est donc un drone grand public de loisirs qui présentent des vulnérabilités importantes. La principale est le réseau Wifi ouvert qui autorise toute personne à portée de signal Wifi à se connecter au drone. La sécurisation du réseau Wifi créée par le drone en utilisant un réseau Wifi WPA2 permettrait de combler cette faille majeure. Dans un second temps, les ports ouverts sur le drone donnent accès à des services sur celui-ci sans authentification de la part de l'utilisateur. Ainsi un accès \textbf{FTP} permettant un accès aux fichiers du drone est laissé à tout utilisateur connecté au réseau Wifi de celui-ci. De plus, faille plus importante, un service \textbf{Telnet} est également accessible fournissant un accès \textbf{root} sur le drone à tout utilisateur connecté au Wifi. Cet accès est une faille majeure car elle permet à un attaquant d'avoir tous les droits sur le drone et ainsi de pouvoir faire ce qu'il veut au niveau du système d'esxploitation de celui-ci. Dans un troisième temps, le contrôle du drone présente également des failles dans la gestion des commandes envoyées par l'utilisateur. En effet, il est possible pour un attaquant de forger de fausses commandes en se faisant passer pour l'utilisateur légitime et ainsi de contrôler le drone. Le drone n'effectue pas d'authentification des commandes qu'il reçoit et ne vérifie donc pas que celle-ci viennent de l'utilisateur légitime. Pour terminer, nous avons démonter qu'il était possible d'exploiter ces différentes failles afin de prendre le contrôle du drone mais elles peuvent également servir réaliser des attaques de type Déni de Service sur celui-ci et ainsi bloquer la connexion entre lui et le client.
\newline La place de plus en plus importante que prennent ces drones de loisir dans l'espace aérien va nécessiter dans le futur que ceux-ci ne présentent plus ce type de failles majeures afin d'empêcher que des attaquants puissent en prendre le contrôle ou  plus simplement puissent réfuter le contrôle de l'aéronef à son utilisateur légitime, et ceici pour des raisons de sécurité vis à vis des autres aéronefs évoluant dans le même espace.
