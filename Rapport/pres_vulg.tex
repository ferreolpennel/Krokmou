\section{Présentation vulgarisée du travail}
\subsection{Premiers pas avec le drone}
Ce projet est centré sur la découverte de vulnérabilités sur l'ARDrone 2.0. Une première mise sous tension du drone nous a permis de découvrir que celui-ci propose un réseau Wifi ouvert, donc non protégé, permettant de se connecter à celui-ci et de le contrôler. Cette observation a grandement orienté notre travail. En effet, nous nous attendions à trouver un réseau Wifi sécurisé qu'il aurait été nécessaire de contourner ou de pénétrer afin d'avoir accès au drone. Cependant, le réseau ouvert nous permet en tant qu'attaquant de nous connecter directement au réseau Wifi créé par le drone. L'ARDrone 2.0 supporte la connexion de plusieurs clients simultanés à son réseau Wifi, toutefois seul le premier client a envoyer des paquets UDP de commande est maître du drone et a la possibilité de le contrôler. En supposant qu'un client légitime est connecté au drone, la question est: que peut faire l'attaquant ?

\subsection{Prise de contrôle du drone}
Une première idée est la prise de contrôle du drone par l'attaquant. Une fois connecté au réseau du drone, l'attaquant n'a qu'à déconnecter l'utilisateur légitime pour devenir le maître du drone. Ainsi en envoyant des messages au drone demandant la déconnexion du client légitime, l'attaquant est capable de reprendre le contrôle du drone une fois celui-ci déconnecté.

\subsection{Injection de commandes}
Dans ce cas, le client légitime reste maître du drone. Toutefois l'attaquant se fait passer pour le client légitime et envoie des messages au drone. Il peut ainsi lui envoyer des instructions que le client légitime n'a jamais envoyé. Par exemple, l'attaquant peut envoyer au drone l'instruction d'atterrir à la place du client légitime.

\subsection{Virus sur le drone}
Ce troisième point exploite une faille importante du drone. En effet, celui-ci offre à tout utilisateur connecté à son réseau Wifi la possibilité d'accéder directement au système 
d'exploitation du drone en ayant tous les droits sur celui-ci et sans authentification et protection. Ainsi, en tant qu'attaquant connecté au réseau Wifi du drone, nous utilisons cet accès au drone pour déposer sur celui-ci un script. Ce script est chargé de copier une image sur toute clé USB connectée au drone ceci afin de démontrer le potentiel de l'attaque. Il est en effet aisé de diffuser n'importe quel virus qui se diffuse par clé USB grâce au drone et ceci sans difficultés particulières.

\subsection{Conclusion}
Ces trois attaques différentes permettent de démontrer les vulnérabilités majeures que présente l'ARDrone 2.0. Une des failles principales de celui-ci est son réseau Wifi non protégé. La protection de celui-ci par du WPA2 permettrait de rendre ces trois différentes attaques beaucoup plus complexes car l'attaquant aurait dans un premier temps besoin de casser le réseau Wifi. De plus, le drone offre des accès non protégés et privilégiés qui sont une faille majeure et permettent à un attaquant une prise de contrôle complète sur le drone.
