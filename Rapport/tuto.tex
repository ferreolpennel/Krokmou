\section{Tutoriel}
Le tutoriel du projet se présente sous la forme d'une application Linux développée en Python permettant d'exploiter les trois types d'attaques présentées précédemment. Disponible à l'adresse suivante: \url{https://github.com/ferreolpennel/Krokmou}, cette application de démonstration est utilisable par tout utilisateur satisafaisant les pré-requis à son installation. L'application, appelée \bsc{Krokmou}, est dédiée à l'ARDrone 2.0 et ne permet des attaques que contre ce type de drone et ce à des fins de démonstration uniquement. Elle permet ainsi de prendre le contrôle d'un drone à la place d'un utilisateur légitime déjà connecté au drone, d'envoyer des commandes pirates au drone sans déconnecter l'utilisateur légitime de celui-ci et de déposer un virus de démonstration sur le drone. Elles illustre ainsi les attaques présentées précédemment et met en relief les failles correspondantes sur ce type de drone. 
