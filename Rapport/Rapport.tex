\documentclass[a4paper,12pt]{report}
\usepackage[utf8]{inputenc}
\usepackage[T1]{fontenc}
\usepackage{lmodern}
\usepackage{eurosym}
\usepackage[french]{babel}
\usepackage{graphicx}

\usepackage{fancyvrb}
\usepackage{wrapfig}


\setlength{\hoffset}{-18pt}
\setlength{\oddsidemargin}{0pt} % Marge gauche sur pages impaires
\setlength{\evensidemargin}{9pt} % Marge gauche sur pages paires
\setlength{\marginparwidth}{54pt} % Largeur de note dans la marge
\setlength{\textwidth}{461pt} % Largeur de la zone de texte (17cm)
\setlength{\voffset}{-18pt} % Bon pour DOS
\setlength{\marginparsep}{7pt} % Séparation de la marge
\setlength{\topmargin}{0pt} % Pas de marge en haut
\setlength{\headheight}{13pt} % Haut de page
\setlength{\headsep}{10pt} % Entre le haut de page et le texte
\setlength{\footskip}{27pt} % Bas de page + séparation
\setlength{\textheight}{708pt} % Hauteur de la zone de texte (25cm)

% Title Page
\title{Rapport de projet long \\ Attaques sur un AR Drone 2.0 Parrot}
\author{Kevyn \bsc{Ledieu}, Ferréol \bsc{Pennel}, Alexis \bsc{Pernot}}

\renewcommand{\thesection}{\arabic{section}}

\begin{document}
\maketitle
\tableofcontents
\newpage

\section{Introduction}
De plus en plus présents autour de nous, les drones de loisirs représentent à la fois des opportunités technologiques et des risques de sécurité. Régulièrement, nous pouvons observer des exemples de drones ayant perturber le trafic aérien par leur présence aux abords d'un aéroport. Ainsi un premier risque de sécurité qu'ils représentent est leur intégration dans l'espace aérien, espace que ces nouveaux aéronefs partagent avec de nombreux autres de toutes les tailles. Aussi des nouvelles règles sont à l'étude afin de réglementer ces activités de loisirs. Toutefois, même une fois réglementée et contrôlée, l'activité des drones de loisir présente un second risque de sécurité lié non plus à la gestion du drone par l'opérateur mais au drone lui-même. En effet, dans la course à l'innovation dans ce domaine porteur qu'est le drone de loisir, les entreprises négligent potentiellement l'aspect sécurité hardware et logicielle de leurs drones. Aussi, ceux-ci peuvent présenter de nombreuses vulnérabilités permettant à un attaquant extérieur de potentiellement prendre le contrôle du drone. Dans ce contexte, nous avons décidé d'étudier un drone grand public proposé par un des leaders du marché, la société \textit{Parrot}. Nous étudierons les différentes vulnérabilités potentiellement présentes sur ce drone et mettront en oeuvre un scénario d'attaque sous forme de tutoriel.

\section{Matériel et Méthode}
\subsection{Matériel: \bsc{AR Drone 2.0}}
Nous allons étudier un \bsc{AR Drone 2.0}. C'est un hélicoptère quadrirotor pilotable via une liaison WiFi au travers d'une application disponible sous iOS, Android, Linux ou Windows. C'est un drone civil principalement destiné au divertissement. Il est équipé de:
\medbreak
\begin{itemize}
    \item un \textit{processeur ARM Cortex A8 32 bits cadencé à 1 GHz}
    \item \textit{1 Go de RAM DDR2 cadencée à 200 MHz}
    \item un \textit{système d'exploitation Linux 2.6.32}
    \item un \textit{module WiFi b/g/n}
    \item un \textit{accéléromètre 3 axes}
    \item un \textit{gyroscope 3 axes}
    \item un \textit{capteur de pression}
    \item un \textit{un magnétomètre 3 axes}
    \item des \textit{capteurs de proximité à ultrasons}
    \item une \textit{caméra vericale QVGA}
    \item un \textit{port USB 2.0}
\end{itemize}


\section{Hypothèse de travail}



\section{Présentation vulgarisée du travail}

\section{Présentation scientifique du travail}

\section{Tutoriel}

\section{Conclusion}




\end{document}
